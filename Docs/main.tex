\documentclass{article}

\usepackage{amsmath, amsthm, amssymb, amsfonts}
\usepackage{thmtools}
\usepackage{graphicx}
\usepackage{setspace}
\usepackage{geometry}
\usepackage{float}
\usepackage[hidelinks]{hyperref}
\usepackage[utf8]{inputenc}
\usepackage[spanish]{babel}
\usepackage{framed}
\usepackage[dvipsnames]{xcolor}
\usepackage{tcolorbox}
\usepackage{tikz}
\usepackage{caption}
\usepackage{longtable}
\usepackage{pdflscape}
\usepackage{svg}
\usepackage{subcaption}
\usepackage{caption}
\usepackage{multirow}
\usepackage{array}
\usepackage{listings}
\usepackage{cancel}


\begin{document}

En el proyecto se utilizan las ecuaciones de Saint Venant para poder modelar el comportamiento del cauldal del río y de su profundidad.

La ecuación de continuidad es:

\begin{equation}
  \label{eq:cont-sv}
  \frac{\partial s_{c} (A + A_{0})}{\partial t} + \frac{\partial Q}{\partial x} - q_{L} = 0
\end{equation}



Donde $s_{c}$ es el coeficiente de sinuosidad, $A$ es el área, $t$ es el tiempo, $Q$ es el cauldal, $x$ es la distancia del río, y, $q_{L}$ es el flujo lateral que se define como:

\[
q_{L} = q_{L1} + q_{L2}
\]

Donde $q_{L1}$ es el sobreflujo lateral y $q_{L2}$ es la filtración del desborde (lo que sale).

Por otro lado, se tiene la ecuación del momentum:

\begin{equation}
  \label{eq:momen-sv}
  \frac{\partial S_{m}Q}{\partial t} + \frac{\partial (\beta Q^{2} / A) }{\partial x} + g A \left(\frac{\partial h_{r}}{\partial x} + S_{f} + S_{ec}  \right) + M_{L} = 0 
\end{equation}



Donde $S_{m}$ es el factor de sinuosidad, $Q$ es el caudal, $t$ es el tiempo, $\beta$ es el coeficiente del momentum para la distribución de la velocidad, $A$ es el área, $x$ es la distancia del río, $g$ es la aceleración de la gravedad, $h_{r}$ es la altura del río, $S_{f}$ es la fricción de la pendiente, $S_{ec}$ es la pérdida por fricción, y, $M_{L}$ es el flujo del momento en los bordes.

Se va a realizar una aproximación numérica mediante el método de diferencias finitas, por lo tanto, las ecuaciones \ref{eq:cont-sv} y \ref{eq:momen-sv} se adaptan a continuación, en primer lugar, la ecuación \ref{eq:cont-sv} de continuidad.

\begin{align*}
  &\frac{\partial s_{c} (A + A_{0})}{\partial t} + \frac{\partial Q}{\partial x} - q_{L} = 0 \\
  &\frac{s_{c} (A + A_{0})_{i}^{j+1} - s_{c}(A + A_{0})_{i}^{j} }{\Delta t} + \frac{Q_{i}^{j} - Q_{i-1}^{j} }{\Delta x} - q_{Li}^{j} = 0   
\end{align*}

\begin{tcolorbox}
\begin{equation}
  \label{eq:cont-sim}
  A_{i}^{j+1} = \frac{\left(q_{Li}^{j} - \frac{Q_{i}^{j} - Q_{i-1}^{j} }{\Delta x}  \right) \Delta t + s_{c} (A + A_{0})_{i}^{j}}{s_{c}} - (A_{0})_{i}^{j+1}
\end{equation}
  
\end{tcolorbox}


Ahora, para la ecuación \ref{eq:momen-sv} de momentum:

\[
\frac{s_{m} (Q)_{i}^{j+1} - s_{m}(Q)_{i}^{j} }{\Delta t} + \frac{(\beta Q^{2}/A)_{i}^{j} - (\beta Q^{2} /A)_{i-1}^{j} }{\Delta x} + gA \left( \frac{(h_{r})_{i}^{j} - (h_{r})_{i-1}^{j}}{\Delta x} + S_{f} + S_{ec}   \right) + M_{L} = 0
\]

Despejando para $Q_{i}^{j+1}$:

\begin{tcolorbox}
\begin{equation}
  \label{eq:momen-sim}
  Q_{i}^{j+1} = \frac{\left[ -M_{L} - g(A_{i}^{j}) \left(\frac{\left(\frac{A}{W}\right)_{i}^{j} - \left(\frac{A}{W}\right)_{i-1}^{j} }{\Delta x} + S_{t} + S_{a}  \right) - \frac{ \left( \frac{\beta Q^{2}}{A} \right)_{i}^{j} - \left( \frac{\beta Q^{2}}{A} \right)_{i-1}^{j}  }{\Delta x}           \right] \Delta t + S_{m} (Q)_{i}^{j}}{s_{m}}
\end{equation}

  
\end{tcolorbox}


Se hizo el cambio de variable $h = \frac{A}{W}$


\end{document}
